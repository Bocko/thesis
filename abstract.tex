\chapter{Abstract}

In the past decades, the microelectronic industry has been following the Moore's law to improve the performance of integrated circuits (IC). However, it will probably be impossible to follow this law in the future due to physical limitations appearing with the miniaturization of the transistors below a certain threshold without innovation. In order to overcome this problem, new technologies have emerged, and among them the 3D-Stacked Integrated Circuits (3D-SIC) have been proposed to keep the Moore's momentum alive. 3D-SICs can bring numerous advantages in the design of future ICs but at the cost of additional design complexity due to their highly combinatorial nature, and the optimization of several conflicting criteria. In this thesis, we present a first study of tools that can help the design of 3D-SICs, using mutiobjective optimization (MOO) and multi-criteria decision aid (MCDA). Our study has targeted one of the main issues in the design of 3D-SICs: the floorplanning. This thesis shows that the use of a multi-criteria paradigm can provide relevant and objective analysis of the problem. This can allow a quick design space exploration and an improvement of the current design flows. Also, with its flexibility, MOO can cope with the multiple degrees of freedom of 3D-SICs, which enables more design possibilities, and can show robustness properties even if the problem is complex. With this methodology, it is possible to give qualitative and quantitative information about a design space, that would not be available with current tools. Finally, applying multi-criteria decision aid would allow designers to make relevant choices in a transparent process.