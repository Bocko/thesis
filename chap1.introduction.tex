\chapter*{Introduction}
\addcontentsline{toc}{chapter}{Introduction}
\fancyhead[RE]{\textbf{Introduction}}
\fancyhead[LO]{} 

\section*{2D architecture, current design flows and their limitations}
In order to continuously improve the performance of integrated circuits (IC), technologists deploy enormous efforts to produce IC manufacturing process that is compelling to follow the well-known Moore's Law (see Figure \ref{fig:mooreslaw}). This empirical law predicts a doubling of the transistors' integration each 18 months and therefore increasing logic capacity of the circuit per unit area. 

\begin{figure}
\begin{center}
\includegraphics[width=0.7\linewidth]{mooreslaw.png}
\end{center}
\vspace{-0.5cm}
\caption{Moore's law \cite{mooreslawpic}}
\label{fig:mooreslaw}
\end{figure}

The improvements of 2D architectures are primarily driven by the reduction of the transistor size. By reducing transistor dimensions, the switching speed is increased thanks to the shorter distance between the source and the drain, implying an improvement of the overall speed of the designs.

However, as the transistor size is decreasing, the observed improvement is also getting smaller. Indeed, a smaller transistor allows higher device density but will slightly increase the total delay (sum of gate and interconnection delays) at the level of the complete circuit.

Also, the total power consumption is increased due to higher leakage and increasing interconnection wire length \cite{5227192}. Figure \ref{fig:delaygateinterconnect} shows the trends in transistor gate delay and interconnect delay with IC fabrication technology where the crossover point represents the interconnect bottleneck. As transistor gate dimensions decrease, designs stretch the limits of the metal interconnection lines that bind the devices, and interconnect performance has become the main limit to IC performance \cite{kirchain2007}.

\begin{figure}
\begin{center}
\includegraphics[width=0.7\linewidth]{delaygateinterconnect}
\end{center}
\caption{Trends in transistor gate delay and interconnect delay with IC fabrication technology \cite{kirchain2007}}
\label{fig:delaygateinterconnect}
\end{figure}

With the miniaturization, quantum effects such as quantum tunnelling will significantly affect how a transistor behave. Indeed, even if a transistor is blocking, current can flow through due to quantum tunnelling such that it will be difficult to control its state and thus the basic working principle of a transistor \cite{1240081}.

In addition to these physical aspects, economical considerations that will hinder the IC evolution beyond 20nm have to be taken into account \cite{5227192,PFF10}.

In order to overcome these limitations, new technologies have been proposed such as the carbon nanotubes \cite{tans1998room}, the nanowire transistors \cite{doi:10.1021/nl025875l}, the single-electron transistors \cite{citeulike:4194929}, but also the 3D-Stacked Integrated Circuits (3D-SIC) proposed by the academic and industrial communities. The latter has been often cited as the most prominent one as it is based on the current technologies and still uses silicon as basis material; 3D-SICs can also allow shorter interconnection lengths, smaller footprint, larger bandwidth, heterogeneous circuits among their main advantages \cite{659500,1652906,981091,4299568}.

Fast evolution of IC manufacturing technologies makes even the design of 2D-ICs a complex and tedious task with the growing number of design choices at the system level (e.g. number and type of functional units and memories, type and topology of the interconnection system, etc.) and physical level (respecting area/timing/power constraints). Using 3D-SICs introduces even more degrees of freedom: number of tiers, choices for manufacturing technology (e.g. full 3D integration, silicon interposer, face-to-face, back-to-face, etc.), 3D partitioning and placement strategies etc. These new degrees of freedom will contribute to the combinatorial explosion of already huge design spaces. Moreover, practice and 2D design experience cannot be fully exploited with 3D technology, since 3D-SICs change considerably the way ICs are implemented. Indeed, physical implementation of ICs involves solving several complex problems and hence work only with approximated solutions.

Current design flows can produce workable solutions after manual definition of the physical constraints as there are no preconceived method that can provide good solutions. Also, they are sequential in nature as certain parameters are fixed at certain stages in the flow, which can lead to locally optimal solutions that are far from global optimums so this requires time consuming (hence, costly) iterative processes to adjust these parameters. Since the 3D technology is even more complex than the 2D, it is necessary to improve the current design flows by developing design exploration \cite{vanderbiest06, PFF10}.

One of the solutions to face this problem is to develop high-level tools which can quickly explore design spaces and give early and reasonably accurate performance estimations based on physical prototyping of the 3D circuits~\cite{PFF10}. In addition, performance estimation/optimization and the selection of the most-suitable solutions usually implies to take several objectives into account (e.g. maximization of the performance, minimization of the cost, minimization of the package size, etc.).

Currently, these high-level design tools can be considered to follow a uni-criterion paradigm. Indeed, they have sequential development steps and each criterion is optimized without considering the impact on other criteria. This can lead to several rollbacks in the design flow since the achievement of the requirements can be time consuming (typical design iterations are measured in weeks). For instance, current tools will only minimize the area of a circuit to reach the timing constraints by solving a 2D place-and-route problem and this will be more complex with 3D-SICs because the system has also to be partitioned.

On the other hand, multi-objective approaches have been developed to optimize all the criteria simultaneously. Designing 3D-SICs inherently implies a huge design space and numerous degrees of freedom and criteria. As it will be shown in Chapter \ref{cha:model}, it appears that modelling a 3D circuit is not an easy task since numerous 3D specificities have to be taken into account. Also, defining and optimizing the criteria is not trivial as they are of heterogeneous nature. Modelling this problem is therefore challenging and these are the reasons why we propose to apply a multi-criteria paradigm for the design of 3D-SICs.

\section*{Research questions}
Multi-objective optimization and multi-criteria decision aid were developed from the need of taking into account several criteria simultaneously. These tools from the operations research field have shown their abilities in solving similar problems in other fields, which also have a large solution space and applying metaheuristics have shown interesting results~\cite{talbi09}.

In this thesis, we will show the applicability of a multi-criteria paradigm for the design of 3D-SICs:
\begin{itemize}
\item How a 3D circuit can be modelled to apply multi-objective optimization?\\
3D-SICs have numerous specificities and taking them into account in a model is not trivial. This will be developed in Section \ref{sec:crit}.

\item How multi-objective optimization can be used to optimize a 3D circuit?\\
The criteria are of heterogeneous nature which can increase the difficulty of optimizing them. This will be presented in Section \ref{sec:nsgaii-implement}.

\item What kind of information can be provided to a designer?\\
With the results that can be obtained with a multi-objective optimization, qualitative and quantitative information that would not be available with current design tools can be provided. This will be illustrated in Section \ref{sec:results}.

\item Can MOO algorithms deal with the complexity of 3D-SICs?\\
With the classical indicators used in the field, good properties of convergence and diversities can be shown for the used algorithm even if the problem is complex. This will be studied in Section \ref{sec:robustness}.

\item How multi-criteria decision aid can exploit these results to assist a designer?\\
After the optimization process, making a choice among the possible designs is not trivial since several criteria are considered simultaneously and MCDA can assist designers facing such situations by modelling their preferences or their constraints. This will be introduced in Chapter \ref{cha:resultsexploit}.

\end{itemize}

\section*{Outline of the manuscript}
In the first chapter, we will take a short overview of the design and manufacturing of 3D-SICs. We will explain the limitations of current design flows if they are applied to the exploration of 3D circuits and present the developments that have been carried out to overcome these problems. We will discuss why they should be improved and introduce how a multi-criteria paradigm can be useful.

In the chapter two, we will present a short overview of the main tools in the MCDA fields where some of the classical methods will be presented.

In the third chapter, we will define the problem we tackle (the 3D partitioning with floorplanning estimation) with the considered criteria. We will then show how a 3D-SIC can be modelled in order to apply multi-objective optimization. Simulations will be run on a case study and show what kind of information can be provided to a designer. The methodology will then be validated with two realistic case studies, a basic one to show what kind of information can be obtained with MOO and a scaled-up one to show the added value of a multi-criteria paradigm compared to a uni-criterion approach.

In the chapter four, we will study the robustness of the methodology and the associated algorithms. We will use classical indicators of the fields to analyse the convergence and diversity properties.

In the fifth chapter, we will explain how the obtained results can be exploited using multi-criteria decision aid. We will discuss on how such a paradigm can be used for designing circuits and what needs to be done in order to integrate it to actual design flows.

Finally, we will conclude on the results of the thesis and express some possible perspectives.

