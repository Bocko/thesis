\chapter{Results exploitation}

\begin{summary}
\lipsum[1]
\end{summary}

\section{Introduction}
In Chapter \ref{cha:model} we have shown how a 3D-SIC can be modelled and the problem we are dealing with in this work, as well as simulation results based on multi-objective optimization. In Chapter \ref{cha:robustness}, we have shown that the methodology can show good convergence and diversity properties even if the problem contains criteria of heterogeneous nature. In this Chapter, we will discuss about how a designer could use these results and take advantage of a multi-criteria oriented methodology in the process of producing a 3D-SIC to, for instance, make a choice among the solutions of the Pareto frontier.

\section{Preference modelling}
As explained in Chapter \ref{cha:rol.mcda}, once a Pareto front has been determined or approximated, the next step is to choose among this set of solutions. One way to help decision makers to make their choice is to model their preferences, for instance with an outranking method. In the scope of this work, we will present the use of the PROMETHEE methodology as it has been developed in our department and has also shown good results in different fields \cite{Beh2010}.

\subsection{PROMETHEE model}
In order to use the PROMETHEE method, the decision maker has to inform about his preferences on the criteria, these being preference functions, indifference and preference thresholds and weights on the criteria. To illustrate this, let us take the case study of Section \ref{} as example.

\section{On the use of a visual tool}


\section{How to pertinently represent multi-criteria information}


\section{On the use of a multi-criteria paradigm in microelectronics design}


\section{Conclusion}