\chapter{Conclusions}
\label{cha:conclusions}
\fancyhead[RE]{\bfseries\leftmark}

In this thesis, we have studied the applicability of multi-objective optimization and multi-criteria decision aid in the context of 3D-stacked integrated circuits design. In the past decades, the electronic industry has been following the Moore’s law to improve the performance of integrated circuits. However, due to physical limitations appearing with the miniaturization of the transistors below a certain threshold, it will probably be impossible to follow this law in the future. In order to overcome this problem, new technologies have emerged, and among them the 3D-Stacked Integrated Circuits have been proposed by the academic and industrial communities. 3D-SICs can bring numerous advantages in the design of future ICs but at the cost of additional design complexity due to their highly combinatorial nature, and requiring the optimization of several conflicting criteria. On the other hand, the multi-criteria approach has been discussed in the literature as a paradigm to adopt for solving numerous similar problems in different industrial fields.

We have therefore proposed to apply multi-objective optimization and multi-criteria decision aid for the design of 3D-SIC. First, we have defined the problem we tackle, the 3D floorplanning, with the considered criteria. Then we have proposed a model for 3D-SIC in order to apply MOO. We have run simulations based on the 3MF MPSoC platform and the obtained results have shown that qualitative and quantitative information, that would not be available with current tools, can be provided to designers. We have also validated the method by using a more realistic case study and have shown that a multi-criteria paradigm has added value compared to a uni-criterion approach, in terms of design space analysis. We have then proved that the methodology and associated algorithms are robust even though the problem of designing 3D-SIC is complex, with criteria of heterogeneous nature. By using classical indicators of the field, we have demonstrated good convergence and diversity properties. Finally, we have shown how the obtained results can be exploited and how multi-criteria decision aid tools can help a designer by providing additional information.

Since the focus of this thesis is to show the applicability of a multi-criteria paradigm to the design of 3D-SIC, there remain several open perspectives. In particular, the accuracy of the model we proposed is yet to be improved. We have shown that MOO can provide quantitative information to designers. Currently, the values have been modelled to respect the order of the solutions in reality. Having a more precise model would allow to propose more realistic quantitative analyses.

Another line of research is the improvement of thermal dissipation criterion. Since this is one of the most critical issues of 3D-SICs, this topic is actually a research field on it own with several developments of thermal-aware floorplanner/partitioner.

In our algorithm, we produce initial solutions randomly. This one of the two possibilities to generate solutions. A further development is to adopt a constructivist approach for building floorplans, for instance with a GRASP-type algorithm. This would require to develop a multi-objective version of GRASP adapted for the design of 3D-SICs. Such an algorithm would allow to dispose of "good" solutions before beginning the design space exploration and may reduce simulation times.

Within the scope of this thesis, we have focused our developments at the logical level of a design flow. With the obtained results, we believe that a multi-criteria methodology can be transposed at other levels such as the architecture level or even the physical level since a uni-criterion paradigm is used at each step. This should give added information to designers and help them facing the growing complexity of ICs.

A particularly important topic that we have not considered in this work is the applications that will run on a platform. This is actually related to hardware/software (HW/SW) co-design which is a research topic on its own. HW/SW co-design aims to match the right software on the right hardware platform in order to take the best out of a design. Considering that HW/SW co-design can be summarized to a multi-criteria combinatorial problem, applying MCDA to that field would already be an improvement. Furthermore, integrating co-design to a performance assessment model will improve it since the estimations will be more precise. For instance, the consumption of a circuit (and its thermal dissipation) will be more accurate as it depends on the application that is run on a platform.


Despite the many questions we open, we still provide an answer to the main research question we posed in the introduction: "Is it possible to apply multi-objective optimization and multi-criteria decision aid to the design of 3D-stacked integrated circuits?". As we have shown, the obtained results provide a first positive answer to this question.