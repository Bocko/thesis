\chapter{Conclusions}
\label{cha:conclusions}
\fancyhead[RE]{\bfseries\leftmark}

In this thesis, we have studied the applicability of multi-objective optimization and multi-criteria decision aid in the context of 3D-stacked integrated circuits design. In the past decades, the electronic industry has been following the Moore’s law to improve the performances of integrated circuits. However, due to physical limitations appearing with the miniaturization of the transistors below a certain threshold, it will probably be impossible to follow this law in the future with the current tools.

In order to overcome this problem, new technologies have emerged, and among them the 3D-Stacked Integrated Circuits have been proposed. 3D-SICs can bring numerous advantages in the design of future ICs but at the cost of additional design complexity due to their highly combinatorial nature, and requiring the optimization of several conflicting criteria. Indeed, 3D circuits, while being based on 2D-ICs, require additional choices such as the number of layers to use or the place-and-route in these tiers. In addition, 3D-SICs bring more challenges in terms of thermal dissipation, cost and design complexity, which illustrate the need to simultaneously optimize multiple objectives. Besides, the multi-criteria approach has been discussed in the literature as a paradigm to adopt for solving numerous similar problems, namely for 2D-ICs place-and-route problems.

\section*{Contributions}

The contributions of this thesis can be summarized as follows:
\begin{itemize}
\item application of a multi-criteria paradigm to the problem of 3D circuits partitioning with estimation of floorplanning
\item proposition of a model considering 5 criteria and degrees of freedom that are not considered with current tools, with simulations for two case studies: a basic MPSoC platform and a scaled-up circuit to show the added value of a more multi-criteria point of view
\item validation of the robustness of the methodology with the analysis of convergence and diversity indicators used in the field
\end{itemize}

We have proposed to apply multi-objective optimization and multi-criteria decision aid for the design of 3D-SICs. First, we have defined the problem, the 3D partitioning with floorplanning estimation by running the complete design flow with simulation of synthesis and place-and-route. This problem consists in finding the three-dimensional geometrical disposition for a circuit (repartition of the components among the layers and their position on each tier) while optimizing multiple objectives. We have proposed a model for the 3D-SICs that includes 5 criteria and degrees of freedom that are not usually considered such as the form factor of the blocks and the functional heterogeneity (blocks with varying size). Indeed, considering form factor can improve the total interconnection length and the heterogeneity has to be taken into account as it is an advantage of 3D-SICs. This constitutes an improvement compared to current tools which, to the best of our knowledge, use a limited set of criteria (usually interconnection length, cost/area and eventually thermal dissipation) and only make trade-off analyses.

%qualitatively détailler
%détailler mcda contribution
%détailler à quelle section ?
Simulations have been performed on a case study based on an MPSoC platform and the obtained results have shown that qualitative and quantitative information, that would not be available with current tools, can be provided to designers. For instance, with the Pareto front that can be obtained after a multi-objective optimization, it is possible to quantify by how much a concession on one criterion can improve another, or where circuits of 1, 2, 3... tiers are located in the design space by quantitatively comparing them to each other with different Pareto fronts. In addition, we have validated the proposed methodology by using a scaled-up case study with functional heterogeneity and have shown that a more global multi-criteria point of view brings added value compared to trade-off analyses that are performed with current tools. Indeed, we have shown that additional information is available to compare the solutions and making a choice among these alternatives is not as trivial a task as it would seem to be when applying a uni-criterion paradigm. Therefore, this enriches the type of information that can be obtained about a design space.

We have then proved that the methodology and associated algorithms are robust even though the problem of designing 3D-SICs is complex, with criteria of heterogeneous nature. By using classical indicators of the field, we have demonstrated good convergence and diversity properties. Indeed, all the values computed for these metrics have shown that the used algorithm does converge towards better solutions at each iteration while maintaining diversity in the Pareto front.

Finally, we have shown how the obtained results can be exploited and how multi-criteria decision aid tools could help a designer by providing additional information. We have presented how preference modelling (with the PROMETHEE methods) and constraint modelling can help a designer when choosing among several alternatives. For instance, it is possible to establish a ranking to make a choice or filter alternatives that do not achieve a constraint level to reduce the number of solutions for the decision process. We have also mentioned a contribution about how to enrich evaluation tables with multi-criteria information in order to highlight compromise solutions and profiles of alternatives and that can also inform a decision maker about the characteristics of the considered problem. % parralèlement au contexte de la thèse, table d'évaluation + lien perspectives, montrer des évaluations = problème --> perspective

\section*{Perspectives}

%respect order (term anglais)
%bullet points
%limite de la précision + chose qu'on connaîtra pas
Since the focus of this thesis is to show the applicability of a multi-criteria para\hyp{}digm to the design of 3D-SICs, there remain several open perspectives. In particular, the accuracy of the model we proposed has yet to be improved. We have shown that MOO can provide quantitative information to designers. Currently, the values have been modelled to respect the ordinal reality of the solutions. Having a more precise model would allow to propose more realistic quantitative analyses.

Currently, simulations have been performed for about an hour on a computer with limited resources. While this has provided interesting results, they can be refined by using a more powerful machine with eventually longer simulation time.

Within the scope of this thesis, we have focused our developments at the logical level of a design flow. The obtained results have shown that added value can be provided about the design space. With the information it can give and thanks to its flexibility, we believe that a multi-criteria methodology can be transposed at other levels such as the architecture level or even the physical level since a uni-criterion paradigm is used at each step. This should also give added information to designers and help them face the growing complexity of producing ICs.

Another line of research is the improvement of the thermal dissipation criterion. In our model, we have considered the peak output power since computing a thermal map can be time consuming. This choice has provided sufficient results for the scope of our work and an improvement could be to develop a method that can quickly compute the thermal dissipation while limiting precision loss. Since this is one of the most critical issues of 3D-SICs, this topic is actually a research field in itself with several developments of thermal-aware partitioners/floorplanners. One can cite the works in \cite{1594713,1112292,1486402}.

In Chapter \ref{cha:resultsexploit} we have presented how multi-criteria decision aid methods can be used to exploit the obtained simulation results. However, as discussed, circuit specifications can not be easily translated into preference models or constraint models since the values and scales used are different and numerous degrees of freedom have to be considered to define a model. In addition, applying MCDA for the design of ICs is not trivial as the industry is not used to this paradigm. This can be improved by studying how MCDA can be adapted to the microelectronic field in order to address its specific needs. %c'est quoi les limites --> évangélisation à faire... --> trop de dégrés de liberté, échelle différente

%peut-on prendre la structure du modèle GRASP? --> étayer pourquoi il faut adapter
In our algorithm, we produce initial solutions randomly. This is one of the two possibilities to generate solutions. A further development is to adopt a constructivist approach for building partitions, for instance with a GRASP-type algorithm \cite{HarSho87a,DBLP:dblp_conf/sccc/ViannaAVA05}, which is based on the successive constructions of a greedy randomized solution (with a mono-objective greedy function) and its improvement through local search. This would require to develop an extended version of GRASP to consider multiple objectives with the related models adapted for the design of 3D-SICs. Such an algorithm would allow to benefit from "good" solutions before beginning the design space exploration and may reduce simulation times.

A particularly important topic that we have not considered in this work is the applications that will run on a platform. This is actually related to hardware/software (HW/SW) co-design which is a research topic in itself. HW/SW co-design aims to match the right software on the right hardware platform in order to take the best out of a design \cite{abdallah2011}. Considering that HW/SW co-design can be summarized to a multi-criteria combinatorial problem, applying MCDA to that field would already be an improvement. Furthermore, integrating co-design to a performance assessment model will improve it since the estimations will be more precise. For instance, the consumption of a circuit (and its thermal dissipation) will be more accurate as it depends on the application that is run on a platform. In overall this would contribute to improving the global design of a circuit.

As a conclusion, we have proposed in this thesis a methodology based on a multi-criteria paradigm to address the problem of 3D partitioning with floorplanning estimation. To this end, we have built a model with criteria and degrees of freedom that are not considered with current tools. We have performed simulations with a NSGA-II algorithm on two different case studies and have shown that a more global multi-criteria point of view does provide added value in terms of information about the design space. This algorithm has been proven to be robust with good convergence and diversity properties based on classical indicators used in the field. Finally, we have shown how multi-criteria decision aid can help a designer when choosing among several solutions and that MCDA can provide additional information to ease this decision process. %Despite the many questions we open, we still provide an answer to the main research question we stated in the introduction: "Is it possible to apply multi-objective optimization and multi-criteria decision aid to the design of 3D-stacked integrated circuits?". As we have shown, the obtained results provide a first positive answer to this question.