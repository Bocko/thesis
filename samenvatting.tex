\chapter[R\'esum\'e (French Summary)]{R\'esum\'e}
\fancyhead[LO]{\bfseries R\'esum\'e}
\fancyhead[RE]{\bfseries R\'esum\'e}

\vspace{-0.5in}

Ces dernières décennies, l'industrie en microélectronique s'est astreinte à suivre la loi de Moore pour améliorer la performance des circuits intégrés (\textit{Integrated Circuit, IC}). Cependant, il sera sans doute impossible de suivre cette loi dans le futur à cause de limitations physiques apparaissant avec la miniaturisation des transistors en-dessous d'un certain seuil si aucune innovatio n'a lieu. Afin de surmonter ce problème, de nouvelles technologies ont émergées, et parmi elles les circuits 3D (\textit{3D-Stacked Integrated Circuit, 3D-SIC}) ont été proposés pour maintenir l'évolution de la loi de Moore. Les 3D-SIC peuvent apporter de nombreux avantages dans le design des futurs IC mais au coût d'une complexité de design accrue étant donné leur nature fortement combinatoire, et l'optimisation de plusieurs critères conflictuels. Dans cette thèse, nous présentons une première étude des outils qui pourraient aider dans le design de 3D-SIC, en utilisant l'optimisation multi-objectifs (\textit{multiobjective optimization, MOO}) et l'aide multicritère à la décision (\textit{multi-criteria decision aid, MCDA}). Notre étude vise l'une des problématiques principales dans le design de 3D-SIC: le partitionnement et le \textit{floorplanning} en tenant compte de plusieurs objectifs. Cette thèse montre que l'utilisation d'un paradigme multicritère peut fournir une analyse pertinente et objective du problème. Cela peut permettre une exploration rapide de l'espace de design et une amélioration des flots de conception actuels étant donné qu'il est possible de fournir des informations qualitatives et quantitatives par rapport à l'espace de design qui ne seraient pas disponibles avec les outils actuels. De même, de par sa flexibilité, la MOO peut tenir compte des multiples degrés de liberté des 3D-SIC, ce qui permet plus de possibilités de design qui ne sont généralement pas prises en compte avec les outils actuels. De plus, les algorithmes développés peuvent montrer des propriétés de robustesse même si le problème est complexe. Enfin, appliquer l'aide multicritère à la décision pourrait permettre aux designers de faire des choix pertinents selon un processus transparent.